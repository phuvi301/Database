% FORMAT AND PACKAGES
% {
\documentclass[a4paper]{article}
\usepackage{a4wide,amssymb,epsfig,latexsym,multicol,array,hhline,fancyhdr}
\usepackage{tcolorbox}
\usepackage{minted}
\usepackage{vntex}
\usepackage{amsmath}
\usepackage{lastpage}
\usepackage[lined,boxed,commentsnumbered]{algorithm2e}
\usepackage{enumerate}
\usepackage{xcolor}
\usepackage{graphicx}							% Standard graphics package
\usepackage{array}
\usepackage{tabularx, caption}
\usepackage{multirow}
\usepackage{multicol}
\usepackage{rotating}
\usepackage{graphics}
\usepackage{geometry}
\usepackage{setspace}
\usepackage{epsfig}
\usepackage{tikz}
\usepackage{xfrac}
\usepackage{bm}
\usepackage{biblatex}
\usepackage[colorlinks]{hyperref}
\setlength{\parskip}{0.2cm}
\newcommand{\cach}{\hspace*{1.5em}\ignorespaces}
% \usepackage[acronym,toc]{glossaries}
% \usepackage[symbols,nogroupskip,nonumberlist]{glossaries-extra}
\usepackage[
 sort=none,% no sorting or indexing required
 abbreviations,% create list of abbreviations
 symbols,% create list of symbols
 stylemods,style=list, % set the default glossary style
 nogroupskip, nonumberlist, nomain
]{glossaries-extra}


% FORMATTING
% {
\DeclareMathOperator{\arccot}{arccot}
\captionsetup[table]{name=Bảng}
\captionsetup[figure]{name=Hình}
\newenvironment{Description}{\list{}{%
    \let\makelabel\descriptionlabel    % this comes from the original description environment
    \setlength{\rightmargin}{\leftmargin}% this comes from the original quote environment
    \setlength{\labelwidth}{0pt}%          this is new
    }}{\endlist}

\addbibresource{citations.bib}
    
\hypersetup{urlcolor=blue,linkcolor=black,citecolor=black,colorlinks=true} 
\usetikzlibrary{arrows,snakes,backgrounds}
\definecolor{mathblue}{RGB}{0,114,188}
% \makeatletter  \def\m@th{\mathsurround\z@\color{mathblue}} \makeatother
% \everymath{\color{mathblue}}
% \setmathfont[Color=000000]{Arial}
%\usepackage{pstcol} 								% PSTricks with the standard color package
\newtheorem{theorem}{{\bf Theorem}}
\newtheorem{property}{{\bf Property}}
\newtheorem{proposition}{{\bf Proposition}}
\newtheorem{corollary}[proposition]{{\bf Corollary}}
\newtheorem{lemma}[proposition]{{\bf Lemma}}

\AtBeginDocument{\renewcommand{\listfigurename}{Danh sách hình ảnh}}
\AtBeginDocument{\renewcommand{\listtablename}{Danh sách bảng biểu}}
\AtBeginDocument{\renewcommand*\contentsname{Mục lục}}
\AtBeginDocument{\renewcommand*\refname{Tài liệu tham khảo}}
%\usepackage{fancyhdr}

\setlength{\headheight}{40pt}
\pagestyle{fancy}
\fancyhead{} % clear all header fields
\fancyhead[L]{
 \begin{tabular}{rl}
    \begin{picture}(25,15)(0,0)
    \put(0,-8){\includegraphics[width=8mm, height=8mm]{Images/hcmut.png}}
    %\put(0,-8){\epsfig{width=10mm,figure=hcmut.eps}}
   \end{picture}&
	%\includegraphics[width=8mm, height=8mm]{hcmut.png} & %
	\begin{tabular}{l}
		\textbf{\bf \ttfamily Trường Đại học Bách Khoa TP.Hồ Chí Minh}\\
		\textbf{\bf \ttfamily Khoa Khoa học và Kỹ thuật máy tính}
	\end{tabular} 	
 \end{tabular}
}
\fancyhead[R]{
	\begin{tabular}{l}
		\tiny \bf \\
		\tiny \bf 
	\end{tabular}  }
\fancyfoot{} % clear all footer fields
\fancyfoot[L]{\scriptsize \ttfamily Báo cáo Bài tập lớn 1 Hệ cơ sở dữ liệu (TN) (CO2014) - HK251 - Năm học 2025 - 2026}
\fancyfoot[R]{\scriptsize \ttfamily Trang {\thepage}/\pageref{LastPage}}
\renewcommand{\headrulewidth}{0.3pt}
\renewcommand{\footrulewidth}{0.3pt}

\setcounter{secnumdepth}{4}
\setcounter{tocdepth}{4}

\makeatletter
\newcounter {subsubsubsection}[subsubsection]
\renewcommand\thesubsubsubsection{\thesubsubsection .\@alph\c@subsubsubsection}
\newcommand\subsubsubsection{\@startsection{subsubsubsection}{4}{\z@}%
                                     {-3.25ex\@plus -1ex \@minus -.2ex}%
                                     {1.5ex \@plus .2ex}%
                                     {\normalfont\normalsize\bfseries}}
\newcommand*\l@subsubsubsection{\@dottedtocline{3}{10.0em}{4.1em}}
\newcommand*{\subsubsubsectionmark}[1]{}
% \def\m@th{\mathsurround\z@\color{mathblue}}
\makeatother
% }
% }

% ACRONYMS & SYMBOLS
% {
% \makeglossaries
\setabbreviationstyle{long-short}
\newabbreviation{csp}{CSP}{Cutting Stock Problem}
\newabbreviation{ffd}{FFD}{First Fit Decreasing}
\newabbreviation{ga}{GA}{Genetic Algorithm}
\newabbreviation{lp}{LP}{Linear Programming}
% \glsnoexpandfields
\glsxtrnewsymbol[description = {Tập hợp số tự nhiên}]{natural}{\ensuremath{\mathbb{N}}}

% }
%
% DOCUMENT
\begin{document}

% TITLE PAGE
\begin{titlepage}
	\begin{center}
		ĐẠI HỌC QUỐC GIA THÀNH PHỐ HỒ CHÍ MINH\\
		TRƯỜNG ĐẠI HỌC BÁCH KHOA\\
		KHOA KHOA HỌC VÀ KỸ THUẬT MÁY TÍNH\\
	\end{center}

	\vspace{1cm}

	\begin{figure}[h!]
		\begin{center}
			\includegraphics[width=3cm]{Images/hcmut.png}
		\end{center}
	\end{figure}

	\vspace{1cm}


	\begin{center}
		\begin{tabular}{c}
			\multicolumn{1}{c}{\textbf{{\Large Hệ cơ sở dữ liệu (TN) (CO2014)}}} \\
			~~                                                                   \\
			\hline
			\\
			\multicolumn{1}{l}{\textbf{{\Large Báo cáo  }}}                       \\
			\\
			\textbf{\textit{{\Huge Bài tập lớn 1 }}}                          \\
			\\
			\hline
		\end{tabular}
	\end{center}

	\begin{table}[h]
		\centering
		\begin{tabular}{rl}
			\hspace{3 cm}\textbf{GVHD}:
			                    & Dương Huỳnh Anh Đức \\[8pt]

                    \textbf{Lớp}: & L05 \\ [8pt]
			\textbf{Sinh viên}: & Lư Chấn Vũ - 2313955 \emph{(Nhóm 7)} \\
			                    & Nguyễn Phú Vinh - 2313922 \emph{(Nhóm 7)}             \\
			                    & Huỳnh Xuân Quốc Việt - 2313891 \emph{(Nhóm 7)}        \\
			                    & Lê Minh Khoa - 2311593 \emph{(Nhóm 7)}                \\
			                    & Lê Minh Trí - 2313593 \emph{(Nhóm 7, \textbf{Leader})}           \\
		\end{tabular}
	\end{table}

	\begin{center}
		{\footnotesize TP. HỒ CHÍ MINH, 09/2025}
	\end{center}
\end{titlepage}

\pagebreak
\tableofcontents

\pagebreak

% Glossaries
% {}
\printunsrtglossary[type={symbols}, title={Danh sách kí hiệu}]
\printunsrtglossary[type={abbreviations}, title={Danh sách từ viết tắt}]
\pagebreak
\listoffigures
\listoftables
\pagebreak
\addcontentsline{toc}{section}{\listfigurename}
\addcontentsline{toc}{section}{\listtablename}

% 

% Member list
\section*{Danh sách thành viên và nhiệm vụ}
\addcontentsline{toc}{section}{Danh sách thành viên và nhiệm vụ}
\begin{center}
	\begin{table}[H]
		\centering
		\begin{tabular}{|c|c|c|l|c|}
			\hline
			\textbf{STT}       & \textbf{Họ và tên}                    & \textbf{MSSV}            & \textbf{Nhiệm vụ} & \textbf{\% hoàn thành} \\
			\hline
			%%%%%Student 1%%%%%%%%%%
			\multirow{3}{*}{1} & \multirow{3}{*}{Lư Chấn Vũ}           & \multirow{3}{*}{2313955} &
			-                  & \multirow{3}{*}{100\%}                                                                                        \\
			                   &                                       &                          & -                 &                        \\
			\hline
			%%%%%Student 2%%%%%%%%%%
			\multirow{3}{*}{2} & \multirow{3}{*}{Nguyễn Phú Vinh}      & \multirow{3}{*}{2313922} &
			-                  & \multirow{3}{*}{100\%}                                                                                        \\
			                   &                                       &                          & -                 &                        \\
			\hline
			%%%%%Student 3%%%%%%%%%%
			\multirow{3}{*}{3} & \multirow{3}{*}{Huỳnh Xuân Quốc Việt} & \multirow{3}{*}{2313891} &
			-                  & \multirow{3}{*}{100\%}                                                                                        \\
			                   &                                       &                          & -                 &                        \\
			\hline
			%%%%%Student 4%%%%%%%%%%
			\multirow{3}{*}{4} & \multirow{3}{*}{Lê Minh Khoa}         & \multirow{3}{*}{2311593} &
			-                  & \multirow{3}{*}{100\%}                                                                                        \\
			                   &                                       &                          & -                 &                        \\
			\hline
			%%%%%Student 5%%%%%%%%%%
			\multirow{3}{*}{5} & \multirow{3}{*}{Lê Minh Trí}          & \multirow{3}{*}{2313593} &
			-                  & \multirow{3}{*}{100\%}                                                                                        \\
			                   &                                       &                          & -                 &                        \\
			\hline
		\end{tabular}
		\caption{\label{table1}Danh sách thành viên và nhiệm vụ}
	\end{table}
\end{center}
\pagebreak
\section*{\begin{center}
	\textbf{Lời mở đầu}
\end{center}}

\cach Trong kỷ nguyên số hiện nay, dữ liệu được xem như là tài sản quý giá của mọi tổ chức và doanh nghiệp. Việc thu thập, quản lý và khai thác dữ liệu một cách hiệu quả đóng vai trò quan trọng trong quá trình vận hành, phát triển dịch vụ cũng như ra quyết định chiến lược. Cơ sở dữ liệu (CSDL) chính là nền tảng cốt lõi giúp đảm bảo cho các hệ thống thông tin hoạt động ổn định, chính xác và an toàn. Một cơ sở dữ liệu được thiết kế khoa học không chỉ hỗ trợ lưu trữ và xử lý khối lượng dữ liệu lớn mà còn giúp tối ưu hiệu suất, duy trì tính toàn vẹn và tạo điều kiện mở rộng hệ thống trong tương lai.  

Trong thực tế, nhiều lĩnh vực ứng dụng công nghệ thông tin đều cần đến các hệ thống cơ sở dữ liệu, điển hình như thương mại điện tử, ngân hàng, giáo dục, và đặc biệt là lĩnh vực giải trí. Một trong những dịch vụ giải trí phổ biến hiện nay là rạp chiếu phim, nơi nhu cầu đặt vé trực tuyến ngày càng gia tăng. Các ứng dụng đặt vé xem phim không chỉ giúp khách hàng dễ dàng lựa chọn phim, suất chiếu, vị trí ghế ngồi và thanh toán trực tuyến, mà còn hỗ trợ nhà quản lý rạp kiểm soát lịch chiếu, doanh thu, khuyến mãi cũng như tình trạng đặt vé theo thời gian thực. Tất cả những chức năng này đều được xây dựng và vận hành dựa trên một hệ thống cơ sở dữ liệu được thiết kế bài bản.  

Trong khuôn khổ môn \textit{Hệ cơ sở dữ liệu} tại Trường Đại học Bách Khoa -- ĐHQG TP.HCM, nhóm chúng em thực hiện đề tài \textbf{``Xây dựng cơ sở dữ liệu cho hệ thống đặt vé xem phim''}. Mục tiêu của đề tài là phân tích yêu cầu nghiệp vụ của hệ thống, xây dựng sơ đồ EERD (Enhanced Entity-Relationship Diagram) để mô hình hóa các thực thể và mối quan hệ, sau đó tiến hành ánh xạ sang mô hình quan hệ để triển khai dưới dạng các bảng trong hệ quản trị cơ sở dữ liệu. Quá trình này không chỉ giúp nhóm củng cố và vận dụng các kiến thức lý thuyết đã học, mà còn rèn luyện kỹ năng phân tích, mô hình hóa và triển khai một cơ sở dữ liệu trong tình huống thực tế.  

Thông qua báo cáo này, nhóm mong muốn trình bày một cách có hệ thống các bước thiết kế cơ sở dữ liệu cho một ứng dụng đặt vé xem phim, từ khâu phân tích đến triển khai. Đây sẽ là nền tảng quan trọng giúp sinh viên tiếp cận gần hơn với thực tiễn, đồng thời làm quen với các yêu cầu về chất lượng, tính chính xác và khả năng mở rộng của một hệ thống cơ sở dữ liệu trong bối cảnh công nghệ ngày nay.  


\pagebreak
\section{Phân tích và mô tả yêu cầu dữ liệu}
\subsection{Tìm hiểu ứng dụng/hệ thống}
\subsubsection{Tên ứng dụng/hệ thống}
\cach Để khảo sát và tham khảo cho việc xây dựng cơ sở dữ liệu ứng dụng đặt vé xem phim, nhóm lựa chọn hệ thống CGV Cinemas Việt Nam làm đối tượng nghiên cứu chính. CGV hiện là một trong những chuỗi rạp chiếu phim lớn tại Việt Nam, cung cấp ứng dụng di động (CGV Cinemas trên App Store và Google Play) cũng như website chính thức tại địa chỉ: $\href{https://www.cgv.vn/.}{CGV}$.


\indent Ứng dụng cho phép người dùng dễ dàng tra cứu thông tin phim đang chiếu, xem trailer, lịch chiếu, chọn rạp, chọn suất chiếu và ghế ngồi trực tiếp trên giao diện. Sau khi đặt vé, khách hàng có thể thanh toán qua nhiều phương thức khác nhau (thẻ ngân hàng, ví điện tử, thẻ thành viên) và nhận vé điện tử dưới dạng mã QR để quét khi vào rạp. Ngoài ra, ứng dụng còn tích hợp hệ thống hội viên với tính năng tích điểm, đổi quà và sử dụng các ưu đãi, khuyến mãi đi kèm. Những chức năng này phản ánh luồng nghiệp vụ cốt lõi của một ứng dụng đặt vé hiện đại, bao gồm: quản lý phim, lịch chiếu, rạp chiếu, ghế ngồi, đơn đặt vé, thanh toán và ưu đãi khách hàng. Việc nghiên cứu CGV giúp định hình rõ các yêu cầu nghiệp vụ cần thiết cho hệ thống đặt vé, từ đó xây dựng mô hình dữ liệu phù hợp.
\subsubsection{Phân tích nghiệp vụ}
\cach Dựa trên khảo sát hệ thống CGV Cinemas, các chức năng chính của ứng dụng đặt vé xem phim có thể phân thành các nhóm: chức năng cho khách hàng, chức năng cho nhân viên, chức năng cho quản trị rạp và các luồng nghiệp vụ cốt lõi.
\\
\\
\cach \textbf{Chức năng chính:}


\indent Đối với khách hàng, hệ thống cung cấp các tính năng như đăng ký/đăng nhập tài khoản, tìm kiếm phim, xem lịch chiếu, đặt vé trực tuyến, thanh toán, quản lý vé và theo dõi khuyến mãi. Với nhân viên, hệ thống hỗ trợ tra cứu thông tin suất chiếu, xác nhận và in vé, quản lý đặt chỗ tại quầy, cũng như hỗ trợ khách hàng trong quá trình sử dụng dịch vụ. Nhóm chức năng dành cho quản trị rạp bao gồm quản lý phim, suất chiếu, phòng chiếu, ghế ngồi, giá vé, chương trình khuyến mãi, cùng với việc thống kê doanh thu và theo dõi tình trạng hoạt động của rạp. Các nhóm chức năng này kết hợp với nhau tạo nên một hệ thống đồng bộ, đáp ứng đầy đủ nhu cầu từ người dùng cuối đến công tác vận hành và quản lý rạp chiếu phim.
\\
\\
\cach \textbf{Luồng nghiệp vụ cốt lõi:}
\begin{enumerate}
    \item Khách hàng mở ứng dụng $\rightarrow$ xem danh sách phim $\rightarrow$ chọn phim.
    \item Chọn rạp $\rightarrow$ chọn ngày giờ $\rightarrow$ chọn ghế $\rightarrow$ chọn combo (nếu có).
    \item Thực hiện thanh toán $\rightarrow$ nhận vé điện tử (QR code).
    \item Đến rạp $\rightarrow$ quét QR code tại cổng $\rightarrow$ vào phòng chiếu.
    \item Sau khi xem phim $\rightarrow$ hệ thống lưu lịch sử, cộng điểm hội viên.
\end{enumerate}

\subsection{Mô tả hệ thống đề xuất}
\subsubsection{Mô tả người dùng và chức năng chính của hệ thống}
\cach Ứng dụng đặt vé xem phim được thiết kế để phục vụ nhu cầu mua vé nhanh, chọn ghế trực quan và quản lý vận hành rạp hiệu quả. Hệ thống gồm các vai trò chính, mỗi vai trò có tập chức năng riêng phục vụ cả trải nghiệm khách hàng và nghiệp vụ rạp.
\\
\\
\cach \textbf{Người dùng của hệ thống}
\begin{itemize}
	\item \textbf{Khách hàng (User, End-user)}: là người dùng cuối truy cập app/web để tra cứu phim, đặt vé và sử dụng dịch vụ tại rạp. Khách hàng có thể là người dùng chưa đăng ký (khách vãng lai) hoặc thành viên (có tài khoản, tích điểm, nhận ưu đãi). Nhu cầu chính: thông tin phim rõ ràng, tìm rạp nhanh, chọn ghế trực quan, thanh toán an toàn, nhận vé điện tử, theo dõi lịch sử và ưu đãi cá nhân.
	\item \textbf{Nhân viên rạp (Staff / Box Office / Gate Concession)}: là nhân viên trực tiếp vận hành tại rạp: bán vé tại quầy, xác thực vé QR ở cổng, xử lý đổi/hủy, quản lý tình trạng ghế và hỗ trợ khách. Họ cần giao diện đơn giản để check-in, huỷ/đổi vé, khóa ghế tạm thời, và truy xuất thông tin đơn hàng nhanh.
	\item \textbf{Quản trị hệ thống (Admin / Manager)}:là người quản lý cấp cao của rạp hoặc chuỗi rạp, chịu trách nhiệm cấu hình hệ thống: thêm phim, lập lịch chiếu, điều chỉnh giá, tạo khuyến mãi, xem báo cáo doanh thu, quản lý tài khoản nhân viên và phân quyền. Họ cần công cụ báo cáo, audit log và cấu hình tích hợp (cổng thanh toán, POS).
\end{itemize}
\cach \textbf{Chức năng chính của hệ thống}
\begin{itemize}
	\item \textbf{Chức năng dành cho khách hàng:} Khách hàng có thể: đăng ký/đăng nhập và quản lý hồ sơ cá nhân; xem danh sách phim, xem chi tiết phim (tóm tắt, thời lượng, thể loại, độ tuổi), và xem trailer; tìm kiếm và lọc phim theo rạp/ngày/thể loại; chọn rạp và suất chiếu; chọn ghế trực quan trên sơ đồ phòng (hiển thị ghế trống/đã bán/không sử dụng); thêm combo đồ ăn/đồ uống vào đơn; áp dụng mã khuyến mãi hoặc sử dụng điểm thành viên khi đặt vé; thanh toán trực tuyến qua thẻ/QR/ví điện tử hoặc thanh toán tại quầy; nhận vé điện tử kèm mã QR và email/xác nhận; xem lịch sử đặt vé, in lại/e-ticket và yêu cầu đổi/hủy theo chính sách; nhận thông báo đẩy về khuyến mãi, thay đổi suất chiếu hoặc nhắc lịch; và đánh giá phim/để lại phản hồi. Mỗi chức năng phải rõ trạng thái (thành công/thất bại) và có thông báo lỗi dễ hiểu.
	\item \textbf{Chức năng dành cho nhân viên rạp:} Nhân viên có thể: check-in khách bằng quét mã QR hoặc nhập mã thủ công; xác thực và huỷ mã QR; thực hiện bán vé trực tiếp tại quầy (tạo đơn, chọn ghế, in vé giấy); quản lý sơ đồ ghế trong ca (khóa/giải phóng ghế, đánh dấu ghế hỏng); xử lý yêu cầu đổi/hủy theo quy định (hoàn tiền, đổi suất); quản lý đơn hàng combo/kho hàng quầy; xem danh sách suất chiếu trong ca và số ghế còn trống; hỗ trợ in lại vé hoặc gửi lại vé điện tử cho khách; và báo cáo vấn đề kỹ thuật hoặc yêu cầu hỗ trợ lên admin. Giao diện dành cho nhân viên cần thao tác nhanh, ít bước, và có kiểm tra phân quyền.
	\item \textbf{Chức năng dành cho quản trị hệ thống:} Admin thực hiện: quản lý phim (thêm, sửa, ngưng chiếu, upload poster/trailer), quản lý rạp và phòng chiếu (thêm rạp, cấu hình phòng, sơ đồ ghế, loại phòng), lập lịch chiếu và điều chỉnh suất (cập nhật giá, format, thời lượng); thiết lập chính sách giá (giá theo loại ghế, khung giờ, ưu đãi), tạo/quản lý chương trình khuyến mãi và mã giảm giá; quản lý người dùng và phân quyền (tạo tài khoản nhân viên, cấp/thu quyền); giám sát đơn hàng và quản lý tài chính (đối chiếu giao dịch, hoàn tiền, quản lý phiếu thu); báo cáo và phân tích (doanh thu theo rạp/phim/suất, tỉ lệ lấp ghế, báo cáo theo khoảng thời gian); cấu hình tích hợp (cổng thanh toán, hệ thống POS, email/SMS gateway); theo dõi nhật ký hoạt động (audit log), backup dữ liệu và cài đặt bảo mật; và quản lý hệ thống (cấu hình bản thử nghiệm, release, monitor). Tất cả hành động quản trị cần có cơ chế kiểm tra truy vết và phân tầng quyền để tránh thao tác trái phép.
\end{itemize}

\subsubsection{Mô tả các kiểu thực thể, các thuộc tính, mối liên kết}
\cach Hệ thống cơ sở dữ liệu đặt vé xem phim quản lý và lưu trữ các thông tin về Người dùng (User), Phim (Movie), Rạp chiếu (Cinema), Phòng chiếu (Room), Suất chiếu (Showtime), Ghế (Seat), Đặt vé (Booking), Thanh toán (Payment), Khuyến mãi (Voucher), Nhân viên (Employee) và các dịch vụ về đồ ăn và thức uống (Food\&Beverage).

Hệ thống đặt vé xem phim cho phép quản lý nhiều rạp cùng lúc. 
Mỗi rạp được gán mã định danh riêng (CinemaID) lưu kèm theo tên (Name) và địa chỉ (Address). 
Mỗi rạp được cấp một số hotline riêng (Hotline) cho phép khách hàng gọi đặt vé online. 

Mỗi rạp quản lý một số lượng phòng chiếu nhất định. 
Mỗi phòng chiếu được quản lý bằng mã phòng (RoomID) riêng biệt giữa các rạp. 
Mỗi phòng được đánh tên riêng (RoomName) và được chia thành nhiều loại phòng (RoomType) với các sức chứa khác nhau (Capacity).

Sức chứa của các phòng là khác nhau nên từng ghế trong mỗi phòng chiếu được quản lý bằng mã riêng biệt (SeatID) đảm bảo phân biệt các ghế trong cùng phòng. 
Mỗi ghế gắn kèm thông tin về hạng ghế (SeatType) cho khách hàng có nhiều sự lựa chọn đa dạng.

Hệ thống cơ sở dữ liệu quản lý và lưu trữ thông tin của khách hàng/người dùng đặt vé. 
Mỗi người dùng được cấp một mã số định danh riêng (UserID) và yêu cầu thông tin về họ và tên (gồm phần tên (FirstName) và phần họ (LastName)), địa chỉ email duy nhất (Email), mật khẩu tài khoản (Password), số điện thoại (Phone) của người dùng. 
Ngoài ra, hệ thống cũng lưu trữ thông tin về ngày sinh (Birthday), giới tính (Gender) cũng như ngày tạo tài khoản (CreateAt) để đáp ứng các dịch vụ mở rộng của hệ thống.

% Người dùng đã có thông tin có thể đặt vé xem phim trực tuyến kết hợp sử dụng các mã khuyến mãi tích lũy được từ hệ thống.
% Mỗi vé được mua được quản lý bằng mã số riêng (BookingID) đồng thời bao gồm thông tin về suất chiếu phim đã đặt (ShowtimeID) và thời gian đã đặt (BookingTime). 
% Đồng thời vé cũng bao gồm giá tiền (TotalPrice) và trạng thái thanh toán hiện tại (Status). Tất nhiên, mỗi người dùng có thể đặt bao nhiêu vé tùy ý.

Người dùng đã có thông tin có thể đặt vé xem phim trực tuyến kết hợp sử dụng các mã khuyến mãi tích lũy được từ hệ thống.
Mỗi vé được quản lý bằng mã số riêng (BookingID) và ghi nhận thời gian đặt (BookingTime). Một đơn vé bao gồm thông tin về suất chiếu phim đã đặt (ShowtimeID), thông tin về vị trí ghế đã chọn (SeatID). 
Tổng giá tiền (TotalPrice) của đơn vé sẽ được tính toán dựa trên các thông tin của vé sau khi đã áp dụng các mã khuyến mãi từ người dùng. 
Tất nhiên, mỗi người dùng có thể đặt bao nhiêu vé tùy ý.

Mỗi suất chiếu được quản lý bằng mã định danh riêng không trùng lặp (ShowtimeID). 
Suất chiếu chứa thông tin về thời gian chiếu (StartTime và EndTime) cũng như giá cơ bản cho một suất (BasePrice). 
Mỗi bộ phim có thể có nhiều suất chiếu khác nhau trong ngày và mỗi suất chiếu được quy định một phòng chiếu cụ thể.

Thông tin về các phim được chiếu được hệ thống lưu trữ chi tiết. 
Cụ thể, mỗi phim được đánh một mã số riêng để quản lý (MovieID), cũng như bao gồm tên phim (Title), thể loại (Genre), thời lượng chiếu (Duration), ngôn ngữ hỗ trợ (Language), độ tuổi giới hạn (AgeRating), mô tả cơ bản về nội dung phim (Description). 
Ngoài ra, mỗi phim còn lưu thêm thông tin về các diễn viên trong phim (Actors), đạo diễn bộ phim (Director), ngày khởi chiếu (ReleaseDate) kèm theo poster (Poster) và trailer khởi chiếu (Trailer). 
Mỗi bộ phim có thể là phần trước hoặc phần sau của một series phim nào đó. 

Hệ thống cũng triển khai các chương trình khuyến mãi nhằm thu hút khách hàng. 
Mỗi chiến dịch khuyến mãi tung ra hàng ngàn voucher khuyến mãi khác nhau. 
Các voucher được quản lý bằng mã voucher nhận thưởng (Promo Code). 
Mỗi voucher có kiểu khuyến mãi riêng (DiscountType) như giảm theo phần trăm hay giảm theo số tiền cùng với giá trị được giảm của vé (DiscountValue), đồng thời cũng yêu cầu các điều kiện áp dụng khác nhau cho từng mã (Condition). 
Mỗi đơn vé được đặt có thể áp dụng nhiều mã khuyến mãi khác nhau (nếu có) để giảm giá vé, đồng thời mỗi mã có thể được sử dụng nhiều lần bởi cho nhiều vé khác nhau. 
Mỗi mã cũng cần có thời gian bắt đầu (StartDate) và kết thúc (EndDate) để đảm bảo mã không bị lạm dụng.

Hệ thống cung cấp các dịch vụ đồ ăn thức uống đi kèm nhằm nâng cao trải nghiệm khách hàng. 
Mỗi món ăn thức uống được quản lý bằng mã riêng (FB ID) và bao gồm tên món (FBName), và giá tiền (Price) cho từng món.
Mỗi món ăn thức uống có thể được đặt kèm trong mỗi đơn vé (nếu có) và mỗi đơn vé có thể đặt nhiều món khác nhau bao gồm chỉ số quản lý số lượng món đã đặt (Num of Items). 

Hệ thống đặt vé xem phim được vận hành bởi đội ngũ nhân viên chuyên nghiệp. 
Trong đó, mọi nhân viên được cấp một mã số nhân viên riêng (EmployeeID), được phân bổ vị trí làm việc tại từng rạp riêng biệt (CinemaID). 
Hệ thống có thể kiểm soát được số lượng nhân viên (Num of Employee) đang làm việc tại mỗi rạp.
Hệ thống cũng cần lưu thông tin về tên nhân viên (Name), ca làm việc (Shift) và lương thưởng (Salary) của từng nhân viên. 
Nhân viên được chia thành hai loại chính là quản lý (Manager) và nhân viên (Staff). 
Với nhân viên, hệ thống lưu trữ về vị trí làm việc (Position) để phân biệt các vai trò khác nhau trong rạp.
Với quản lý, hệ thống lưu trữ số lượng các nhân viên (EmployeeID) mà họ quản lý trực tiếp. 

Hệ thống cho phép người dùng thanh toán bằng nhiều hình thức khác nhau và lưu trữ thông tin thanh toán. 
Mỗi thanh toán/hóa đơn có mã hóa đơn riêng (PaymentID) và thuộc về một vé duy nhất (BookingID). 
Mỗi thanh toán có thể là giao dịch qua thẻ tín dụng, ví điện tử hay trực tiếp bằng tiền mặt. 
Đối với các thanh toán trực tuyến bằng, hệ thống lưu trữ thông tin về mã giao dịch (PaymentCode) do bên thứ ba cung cấp và thông tin về số tài khoản (Card Number) và ngân hàng xử lý giao dịch (Bank Name). 
Đối với các thành toán trực tiếp tại quầy, hệ thống lưu trữ thông tin về tên nhân viên thu ngân (StaffName). 
Trạng thái thanh toán (Status) cũng cần được lưu cho phép hệ thống kiểm tra giao dịch hoàn tất hay chưa. 
Khi giao dịch hoàn tất, hệ thống cũng cần ghi nhận thời gian thực hiện giao dịch (PaymentTime), đảm bảo tính minh bạch cho hệ thống, nâng cao sự tín nhiệm với khách hàng.

% Vinh lỏ vẫn lỏ

\subsection{Mô tả các ràng buộc ngữ nghĩa}
\begin{itemize}
	\item Sức chứa (Capacity) của mỗi phòng chiếu (Room) phải lớn hơn 0.
	\item Số ghế (SeatNumber) trong cùng phòng không được trùng nhau nếu cùng thuộc một hàng (Row).
	\item Số điện thoại (Phone) của khách hàng phải là chuỗi có 10 chữ số.
	\item Thời gian bắt đầu (StartTime) của suất chiếu (Showtime) phải sớm hơn thời gian kết thúc (EndTime) của nó.
	\item Các suất chiếu trong cùng một phòng không được chồng lấn StartTime và EndTime.
	\item Số tiền thanh toán (Amout) phải bằng hoặc nhỏ hơn tổng giá vé (TotalPrice).
	\item Thời gian đặt vé (BookingTime) phải sớm hơn thời gian kết thúc (EndTime) của suất chiếu.
	\item Thời gian bắt đầu (StartDate) của mã khuyến mãi (Voucher) phải sớm hơn thời gian kết thúc (EndDate) của nó.
	\item Độ tuổi giới hạn (AgeRating) của mỗi bộ phim (Movie) không thể là số âm.
	\item Mật khẩu đăng nhập (Password) của cả nhân viên (Employee) và người dùng (User) phải là chuỗi có ít nhất 8 kí tự, trong đó phải có ít nhất 1 kí tự chữ hoa, ít nhất 1 kí tự số, ít nhất 1 kí tự đặc biệt thuộc tập ($@$, $\mathdollar$, $\#$, $\%$, !, $\&$).
\end{itemize}
\section{Thiết kế EERD}
\section{Ánh xạ lược đồ EERD sang lược đồ CSDL}

\end{document}
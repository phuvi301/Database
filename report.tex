% FORMAT AND PACKAGES
% {
\documentclass[a4paper]{article}
\usepackage{a4wide,amssymb,epsfig,latexsym,multicol,array,hhline,fancyhdr}
\usepackage{tcolorbox}
\usepackage{minted}
\usepackage{vntex}
\usepackage{amsmath}
\usepackage{lastpage}
\usepackage[lined,boxed,commentsnumbered]{algorithm2e}
\usepackage{enumerate}
\usepackage{xcolor}
\usepackage{graphicx}							% Standard graphics package
\usepackage{array}
\usepackage{tabularx, caption}
\usepackage{multirow}
\usepackage{multicol}
\usepackage{rotating}
\usepackage{graphics}
\usepackage{geometry}
\usepackage{setspace}
\usepackage{epsfig}
\usepackage{tikz}
\usepackage{xfrac}
\usepackage{bm}
\usepackage{biblatex}
\usepackage[colorlinks]{hyperref}
\newcommand{\cach}{\hspace*{1.5em}\ignorespaces}
% \usepackage[acronym,toc]{glossaries}
% \usepackage[symbols,nogroupskip,nonumberlist]{glossaries-extra}
\usepackage[
 sort=none,% no sorting or indexing required
 abbreviations,% create list of abbreviations
 symbols,% create list of symbols
 stylemods,style=list, % set the default glossary style
 nogroupskip, nonumberlist, nomain
]{glossaries-extra}


% FORMATTING
% {
\DeclareMathOperator{\arccot}{arccot}
\captionsetup[table]{name=Bảng}
\captionsetup[figure]{name=Hình}
\newenvironment{Description}{\list{}{%
    \let\makelabel\descriptionlabel    % this comes from the original description environment
    \setlength{\rightmargin}{\leftmargin}% this comes from the original quote environment
    \setlength{\labelwidth}{0pt}%          this is new
    }}{\endlist}

\addbibresource{citations.bib}
    
\hypersetup{urlcolor=blue,linkcolor=black,citecolor=black,colorlinks=true} 
\usetikzlibrary{arrows,snakes,backgrounds}
\definecolor{mathblue}{RGB}{0,114,188}
% \makeatletter  \def\m@th{\mathsurround\z@\color{mathblue}} \makeatother
% \everymath{\color{mathblue}}
% \setmathfont[Color=000000]{Arial}
%\usepackage{pstcol} 								% PSTricks with the standard color package
\newtheorem{theorem}{{\bf Theorem}}
\newtheorem{property}{{\bf Property}}
\newtheorem{proposition}{{\bf Proposition}}
\newtheorem{corollary}[proposition]{{\bf Corollary}}
\newtheorem{lemma}[proposition]{{\bf Lemma}}

\AtBeginDocument{\renewcommand{\listfigurename}{Danh sách hình ảnh}}
\AtBeginDocument{\renewcommand{\listtablename}{Danh sách bảng biểu}}
\AtBeginDocument{\renewcommand*\contentsname{Mục lục}}
\AtBeginDocument{\renewcommand*\refname{Tài liệu tham khảo}}
%\usepackage{fancyhdr}

\setlength{\headheight}{40pt}
\pagestyle{fancy}
\fancyhead{} % clear all header fields
\fancyhead[L]{
 \begin{tabular}{rl}
    \begin{picture}(25,15)(0,0)
    \put(0,-8){\includegraphics[width=8mm, height=8mm]{Images/hcmut.png}}
    %\put(0,-8){\epsfig{width=10mm,figure=hcmut.eps}}
   \end{picture}&
	%\includegraphics[width=8mm, height=8mm]{hcmut.png} & %
	\begin{tabular}{l}
		\textbf{\bf \ttfamily Trường Đại học Bách Khoa TP.Hồ Chí Minh}\\
		\textbf{\bf \ttfamily Khoa Khoa học và Kỹ thuật máy tính}
	\end{tabular} 	
 \end{tabular}
}
\fancyhead[R]{
	\begin{tabular}{l}
		\tiny \bf \\
		\tiny \bf 
	\end{tabular}  }
\fancyfoot{} % clear all footer fields
\fancyfoot[L]{\scriptsize \ttfamily Báo cáo Bài tập lớn 1 Hệ cơ sở dữ liệu (TN) (CO2014) - HK251 - Năm học 2025 - 2026}
\fancyfoot[R]{\scriptsize \ttfamily Trang {\thepage}/\pageref{LastPage}}
\renewcommand{\headrulewidth}{0.3pt}
\renewcommand{\footrulewidth}{0.3pt}

\setcounter{secnumdepth}{4}
\setcounter{tocdepth}{4}

\makeatletter
\newcounter {subsubsubsection}[subsubsection]
\renewcommand\thesubsubsubsection{\thesubsubsection .\@alph\c@subsubsubsection}
\newcommand\subsubsubsection{\@startsection{subsubsubsection}{4}{\z@}%
                                     {-3.25ex\@plus -1ex \@minus -.2ex}%
                                     {1.5ex \@plus .2ex}%
                                     {\normalfont\normalsize\bfseries}}
\newcommand*\l@subsubsubsection{\@dottedtocline{3}{10.0em}{4.1em}}
\newcommand*{\subsubsubsectionmark}[1]{}
% \def\m@th{\mathsurround\z@\color{mathblue}}
\makeatother
% }
% }

% ACRONYMS & SYMBOLS
% {
% \makeglossaries
\setabbreviationstyle{long-short}
\newabbreviation{csp}{CSP}{Cutting Stock Problem}
\newabbreviation{ffd}{FFD}{First Fit Decreasing}
\newabbreviation{ga}{GA}{Genetic Algorithm}
\newabbreviation{lp}{LP}{Linear Programming}
% \glsnoexpandfields
\glsxtrnewsymbol[description = {Tập hợp số tự nhiên}]{natural}{\ensuremath{\mathbb{N}}}

% }
%
% DOCUMENT
\begin{document}

% TITLE PAGE
\begin{titlepage}
	\begin{center}
		ĐẠI HỌC QUỐC GIA THÀNH PHỐ HỒ CHÍ MINH\\
		TRƯỜNG ĐẠI HỌC BÁCH KHOA\\
		KHOA KHOA HỌC VÀ KỸ THUẬT MÁY TÍNH\\
	\end{center}

	\vspace{1cm}

	\begin{figure}[h!]
		\begin{center}
			\includegraphics[width=3cm]{Images/hcmut.png}
		\end{center}
	\end{figure}

	\vspace{1cm}


	\begin{center}
		\begin{tabular}{c}
			\multicolumn{1}{c}{\textbf{{\Large Hệ cơ sở dữ liệu (TN) (CO2014)}}} \\
			~~                                                                   \\
			\hline
			\\
			\multicolumn{1}{l}{\textbf{{\Large Báo cáo  }}}                       \\
			\\
			\textbf{\textit{{\Huge Bài tập lớn 1 }}}                          \\
			\\
			\hline
		\end{tabular}
	\end{center}

	\begin{table}[h]
		\centering
		\begin{tabular}{rl}
			\hspace{3 cm}\textbf{GVHD}:
			                    & Dương Huỳnh Anh Đức \\[8pt]

                    \textbf{Lớp}: & L05 \\ [8pt]
			\textbf{Sinh viên}: & Lư Chấn Vũ - 2313955 \emph{(Nhóm 7)} \\
			                    & Nguyễn Phú Vinh - 2313922 \emph{(Nhóm 7)}             \\
			                    & Huỳnh Xuân Quốc Việt - 2313891 \emph{(Nhóm 7)}        \\
			                    & Lê Minh Khoa - 2311593 \emph{(Nhóm 7)}                \\
			                    & Lê Minh Trí - 2313593 \emph{(Nhóm 7, \textbf{Leader})}           \\
		\end{tabular}
	\end{table}

	\begin{center}
		{\footnotesize TP. HỒ CHÍ MINH, 09/2025}
	\end{center}
\end{titlepage}

\pagebreak
\tableofcontents

\pagebreak

% Glossaries
% {}
\printunsrtglossary[type={symbols}, title={Danh sách kí hiệu}]
\printunsrtglossary[type={abbreviations}, title={Danh sách từ viết tắt}]
\pagebreak
\listoffigures
\listoftables
\pagebreak
\addcontentsline{toc}{section}{\listfigurename}
\addcontentsline{toc}{section}{\listtablename}

% 

% Member list
\section*{Danh sách thành viên và nhiệm vụ}
\addcontentsline{toc}{section}{Danh sách thành viên và nhiệm vụ}
\begin{center}
	\begin{table}[H]
		\centering
		\begin{tabular}{|c|c|c|l|c|}
			\hline
			\textbf{STT}       & \textbf{Họ và tên}                    & \textbf{MSSV}            & \textbf{Nhiệm vụ} & \textbf{\% hoàn thành} \\
			\hline
			%%%%%Student 1%%%%%%%%%%
			\multirow{3}{*}{1} & \multirow{3}{*}{Lư Chấn Vũ}           & \multirow{3}{*}{2313955} &
			-                  & \multirow{3}{*}{100\%}                                                                                        \\
			                   &                                       &                          & -                 &                        \\
			\hline
			%%%%%Student 2%%%%%%%%%%
			\multirow{3}{*}{2} & \multirow{3}{*}{Nguyễn Phú Vinh}      & \multirow{3}{*}{2313922} &
			-                  & \multirow{3}{*}{100\%}                                                                                        \\
			                   &                                       &                          & -                 &                        \\
			\hline
			%%%%%Student 3%%%%%%%%%%
			\multirow{3}{*}{3} & \multirow{3}{*}{Huỳnh Xuân Quốc Việt} & \multirow{3}{*}{2313891} &
			-                  & \multirow{3}{*}{100\%}                                                                                        \\
			                   &                                       &                          & -                 &                        \\
			\hline
			%%%%%Student 4%%%%%%%%%%
			\multirow{3}{*}{4} & \multirow{3}{*}{Lê Minh Khoa}         & \multirow{3}{*}{2311593} &
			-                  & \multirow{3}{*}{100\%}                                                                                        \\
			                   &                                       &                          & -                 &                        \\
			\hline
			%%%%%Student 5%%%%%%%%%%
			\multirow{3}{*}{5} & \multirow{3}{*}{Lê Minh Trí}          & \multirow{3}{*}{2313593} &
			-                  & \multirow{3}{*}{100\%}                                                                                        \\
			                   &                                       &                          & -                 &                        \\
			\hline
		\end{tabular}
		\caption{\label{table1}Danh sách thành viên và nhiệm vụ}
	\end{table}
\end{center}
\pagebreak
\section*{\begin{center}
	\textbf{Lời mở đầu}
\end{center}}

\cach Trong kỷ nguyên số hiện nay, dữ liệu được xem như là tài sản quý giá của mọi tổ chức và doanh nghiệp. Việc thu thập, quản lý và khai thác dữ liệu một cách hiệu quả đóng vai trò quan trọng trong quá trình vận hành, phát triển dịch vụ cũng như ra quyết định chiến lược. Cơ sở dữ liệu (CSDL) chính là nền tảng cốt lõi giúp đảm bảo cho các hệ thống thông tin hoạt động ổn định, chính xác và an toàn. Một cơ sở dữ liệu được thiết kế khoa học không chỉ hỗ trợ lưu trữ và xử lý khối lượng dữ liệu lớn mà còn giúp tối ưu hiệu suất, duy trì tính toàn vẹn và tạo điều kiện mở rộng hệ thống trong tương lai.  

\vspace{0.2cm}

\indent Trong thực tế, nhiều lĩnh vực ứng dụng công nghệ thông tin đều cần đến các hệ thống cơ sở dữ liệu, điển hình như thương mại điện tử, ngân hàng, giáo dục, và đặc biệt là lĩnh vực giải trí. Một trong những dịch vụ giải trí phổ biến hiện nay là rạp chiếu phim, nơi nhu cầu đặt vé trực tuyến ngày càng gia tăng. Các ứng dụng đặt vé xem phim không chỉ giúp khách hàng dễ dàng lựa chọn phim, suất chiếu, vị trí ghế ngồi và thanh toán trực tuyến, mà còn hỗ trợ nhà quản lý rạp kiểm soát lịch chiếu, doanh thu, khuyến mãi cũng như tình trạng đặt vé theo thời gian thực. Tất cả những chức năng này đều được xây dựng và vận hành dựa trên một hệ thống cơ sở dữ liệu được thiết kế bài bản.  

\vspace{0.2cm}

\indent Trong khuôn khổ môn \textit{Hệ cơ sở dữ liệu} tại Trường Đại học Bách Khoa -- ĐHQG TP.HCM, nhóm chúng em thực hiện đề tài \textbf{``Xây dựng cơ sở dữ liệu cho hệ thống đặt vé xem phim''}. Mục tiêu của đề tài là phân tích yêu cầu nghiệp vụ của hệ thống, xây dựng sơ đồ EERD (Enhanced Entity-Relationship Diagram) để mô hình hóa các thực thể và mối quan hệ, sau đó tiến hành ánh xạ sang mô hình quan hệ để triển khai dưới dạng các bảng trong hệ quản trị cơ sở dữ liệu. Quá trình này không chỉ giúp nhóm củng cố và vận dụng các kiến thức lý thuyết đã học, mà còn rèn luyện kỹ năng phân tích, mô hình hóa và triển khai một cơ sở dữ liệu trong tình huống thực tế.  

\vspace{0.2cm}

\indent Thông qua báo cáo này, nhóm mong muốn trình bày một cách có hệ thống các bước thiết kế cơ sở dữ liệu cho một ứng dụng đặt vé xem phim, từ khâu phân tích đến triển khai. Đây sẽ là nền tảng quan trọng giúp sinh viên tiếp cận gần hơn với thực tiễn, đồng thời làm quen với các yêu cầu về chất lượng, tính chính xác và khả năng mở rộng của một hệ thống cơ sở dữ liệu trong bối cảnh công nghệ ngày nay.  


\pagebreak
\section{Phân tích và mô tả yêu cầu dữ liệu}
\subsection{Tìm hiểu ứng dụng/hệ thống}
\subsubsection{Tên ứng dụng/hệ thống}
\cach Để khảo sát và tham khảo cho việc xây dựng cơ sở dữ liệu ứng dụng đặt vé xem phim, nhóm lựa chọn hệ thống CGV Cinemas Việt Nam làm đối tượng nghiên cứu chính. CGV hiện là một trong những chuỗi rạp chiếu phim lớn tại Việt Nam, cung cấp ứng dụng di động (CGV Cinemas trên App Store và Google Play) cũng như website chính thức tại địa chỉ: $\href{https://www.cgv.vn/.}{CGV}$.

\vspace{0.2cm}

\indent Ứng dụng cho phép người dùng dễ dàng tra cứu thông tin phim đang chiếu, xem trailer, lịch chiếu, chọn rạp, chọn suất chiếu và ghế ngồi trực tiếp trên giao diện. Sau khi đặt vé, khách hàng có thể thanh toán qua nhiều phương thức khác nhau (thẻ ngân hàng, ví điện tử, thẻ thành viên) và nhận vé điện tử dưới dạng mã QR để quét khi vào rạp. Ngoài ra, ứng dụng còn tích hợp hệ thống hội viên với tính năng tích điểm, đổi quà và sử dụng các ưu đãi, khuyến mãi đi kèm. Những chức năng này phản ánh luồng nghiệp vụ cốt lõi của một ứng dụng đặt vé hiện đại, bao gồm: quản lý phim, lịch chiếu, rạp chiếu, ghế ngồi, đơn đặt vé, thanh toán và ưu đãi khách hàng. Việc nghiên cứu CGV giúp định hình rõ các yêu cầu nghiệp vụ cần thiết cho hệ thống đặt vé, từ đó xây dựng mô hình dữ liệu phù hợp.
\subsubsection{Phân tích nghiệp vụ}
\cach Dựa trên khảo sát hệ thống CGV Cinemas, các chức năng chính của ứng dụng đặt vé xem phim có thể phân thành các nhóm: chức năng cho khách hàng, chức năng cho nhân viên, chức năng cho quản trị rạp và các luồng nghiệp vụ cốt lõi.
\\
\\
\cach \textbf{Chức năng chính:}

\vspace{0.2cm}

\indent Đối với khách hàng, hệ thống cung cấp các tính năng như đăng ký/đăng nhập tài khoản, tìm kiếm phim, xem lịch chiếu, đặt vé trực tuyến, thanh toán, quản lý vé và theo dõi khuyến mãi. Với nhân viên, hệ thống hỗ trợ tra cứu thông tin suất chiếu, xác nhận và in vé, quản lý đặt chỗ tại quầy, cũng như hỗ trợ khách hàng trong quá trình sử dụng dịch vụ. Nhóm chức năng dành cho quản trị rạp bao gồm quản lý phim, suất chiếu, phòng chiếu, ghế ngồi, giá vé, chương trình khuyến mãi, cùng với việc thống kê doanh thu và theo dõi tình trạng hoạt động của rạp. Các nhóm chức năng này kết hợp với nhau tạo nên một hệ thống đồng bộ, đáp ứng đầy đủ nhu cầu từ người dùng cuối đến công tác vận hành và quản lý rạp chiếu phim.
\\
\\
\cach \textbf{Luồng nghiệp vụ cốt lõi:}
\begin{enumerate}
    \item Khách hàng mở ứng dụng $\rightarrow$ xem danh sách phim $\rightarrow$ chọn phim.
    \item Chọn rạp $\rightarrow$ chọn ngày giờ $\rightarrow$ chọn ghế $\rightarrow$ chọn combo (nếu có).
    \item Thực hiện thanh toán $\rightarrow$ nhận vé điện tử (QR code).
    \item Đến rạp $\rightarrow$ quét QR code tại cổng $\rightarrow$ vào phòng chiếu.
    \item Sau khi xem phim $\rightarrow$ hệ thống lưu lịch sử, cộng điểm hội viên.
\end{enumerate}

\subsection{Mô tả hệ thống đề xuất}
\subsubsection{Mô tả người dùng và chức năng chính của hệ thống}
\cach Ứng dụng đặt vé xem phim được thiết kế để phục vụ nhu cầu mua vé nhanh, chọn ghế trực quan và quản lý vận hành rạp hiệu quả. Hệ thống gồm các vai trò chính, mỗi vai trò có tập chức năng riêng phục vụ cả trải nghiệm khách hàng và nghiệp vụ rạp.
\\
\\
\cach \textbf{Người dùng của hệ thống}
\begin{itemize}
	\item \textbf{Khách hàng (User, End-user)}: là người dùng cuối truy cập app/web để tra cứu phim, đặt vé và sử dụng dịch vụ tại rạp. Khách hàng có thể là người dùng chưa đăng ký (khách vãng lai) hoặc thành viên (có tài khoản, tích điểm, nhận ưu đãi). Nhu cầu chính: thông tin phim rõ ràng, tìm rạp nhanh, chọn ghế trực quan, thanh toán an toàn, nhận vé điện tử, theo dõi lịch sử và ưu đãi cá nhân.
	\item \textbf{Nhân viên rạp (Staff / Box Office / Gate Concession)}: là nhân viên trực tiếp vận hành tại rạp: bán vé tại quầy, xác thực vé QR ở cổng, xử lý đổi/hủy, quản lý tình trạng ghế và hỗ trợ khách. Họ cần giao diện đơn giản để check-in, huỷ/đổi vé, khóa ghế tạm thời, và truy xuất thông tin đơn hàng nhanh.
	\item \textbf{Quản trị hệ thống (Admin / Manager)}:là người quản lý cấp cao của rạp hoặc chuỗi rạp, chịu trách nhiệm cấu hình hệ thống: thêm phim, lập lịch chiếu, điều chỉnh giá, tạo khuyến mãi, xem báo cáo doanh thu, quản lý tài khoản nhân viên và phân quyền. Họ cần công cụ báo cáo, audit log và cấu hình tích hợp (cổng thanh toán, POS).
\end{itemize}
\cach \textbf{Chức năng chính của hệ thống}
\begin{itemize}
	\item \textbf{Chức năng dành cho khách hàng:} Khách hàng có thể: đăng ký/đăng nhập và quản lý hồ sơ cá nhân; xem danh sách phim, xem chi tiết phim (tóm tắt, thời lượng, thể loại, độ tuổi), và xem trailer; tìm kiếm và lọc phim theo rạp/ngày/thể loại; chọn rạp và suất chiếu; chọn ghế trực quan trên sơ đồ phòng (hiển thị ghế trống/đã bán/không sử dụng); thêm combo đồ ăn/đồ uống vào đơn; áp dụng mã khuyến mãi hoặc sử dụng điểm thành viên khi đặt vé; thanh toán trực tuyến qua thẻ/QR/ví điện tử hoặc thanh toán tại quầy; nhận vé điện tử kèm mã QR và email/xác nhận; xem lịch sử đặt vé, in lại/e-ticket và yêu cầu đổi/hủy theo chính sách; nhận thông báo đẩy về khuyến mãi, thay đổi suất chiếu hoặc nhắc lịch; và đánh giá phim/để lại phản hồi. Mỗi chức năng phải rõ trạng thái (thành công/thất bại) và có thông báo lỗi dễ hiểu.
	\item \textbf{Chức năng dành cho nhân viên rạp:} Nhân viên có thể: check-in khách bằng quét mã QR hoặc nhập mã thủ công; xác thực và huỷ mã QR; thực hiện bán vé trực tiếp tại quầy (tạo đơn, chọn ghế, in vé giấy); quản lý sơ đồ ghế trong ca (khóa/giải phóng ghế, đánh dấu ghế hỏng); xử lý yêu cầu đổi/hủy theo quy định (hoàn tiền, đổi suất); quản lý đơn hàng combo/kho hàng quầy; xem danh sách suất chiếu trong ca và số ghế còn trống; hỗ trợ in lại vé hoặc gửi lại vé điện tử cho khách; và báo cáo vấn đề kỹ thuật hoặc yêu cầu hỗ trợ lên admin. Giao diện dành cho nhân viên cần thao tác nhanh, ít bước, và có kiểm tra phân quyền.
	\item \textbf{Chức năng dành cho quản trị hệ thống:} Admin thực hiện: quản lý phim (thêm, sửa, ngưng chiếu, upload poster/trailer), quản lý rạp và phòng chiếu (thêm rạp, cấu hình phòng, sơ đồ ghế, loại phòng), lập lịch chiếu và điều chỉnh suất (cập nhật giá, format, thời lượng); thiết lập chính sách giá (giá theo loại ghế, khung giờ, ưu đãi), tạo/quản lý chương trình khuyến mãi và mã giảm giá; quản lý người dùng và phân quyền (tạo tài khoản nhân viên, cấp/thu quyền); giám sát đơn hàng và quản lý tài chính (đối chiếu giao dịch, hoàn tiền, quản lý phiếu thu); báo cáo và phân tích (doanh thu theo rạp/phim/suất, tỉ lệ lấp ghế, báo cáo theo khoảng thời gian); cấu hình tích hợp (cổng thanh toán, hệ thống POS, email/SMS gateway); theo dõi nhật ký hoạt động (audit log), backup dữ liệu và cài đặt bảo mật; và quản lý hệ thống (cấu hình bản thử nghiệm, release, monitor). Tất cả hành động quản trị cần có cơ chế kiểm tra truy vết và phân tầng quyền để tránh thao tác trái phép.
\end{itemize}

\subsubsection{Mô tả các kiểu thực thể, các thuộc tính, mối liên kết}
\subsection{Mô tả các ràng buộc ngữ nghĩa}
\section{Thiết kế EERD}
\section{Ánh xạ lược đồ EERD sang lược đồ CSDL}

\end{document}